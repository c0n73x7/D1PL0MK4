\chapter*{Závěr}
\addcontentsline{toc}{chapter}{Závěr}

V prvních čtyřech kapitolách je shrnuta teorie, která se dále využívá při popisu úloh kombinatorické optimalizace.

V kapitole o Shannonově kapacitě se věnujeme Lovászově $\vartheta$ funkci a jejímu využití. Byly naimplementovány a následně porovnány dva modely s přesnou hodnotou pro liché kružnice. Jednodušší model, který dává rovnou hodnotu $\vartheta$ funkce byl přesnější. U druhého modelu, ze kterého dostaneme hodnotu $1/\sqrt{\vartheta}$, byly patrné chyby už pro malé grafy. Problémem je, že úloha není ve standardním tvaru a frameworky typu Mosek jsou optimalizovány hlavně na práci s úlohami ve standardním tvaru. Dále byl puštěn výpočet pro vylepšení dolního odhadu Shannonovy kapacity kružnice $C_7$, ale nezávislá množina, která by odhad vylepšila se nenašla.

Pro úlohu MAX $k$-CUT byly pro účely testování implementovány čtyři algoritmy a pro úlohu kapacitního MAX $k$-CUTu další dva. U experimentu pro úlohu MAX $3$-CUT vyšel nejhůře algoritmus~\ref{alg:gw-max-3-cut}, který má nejlepší aproximační poměr, což je to dáno složitostí modelu založeného na komplexním semidefinitním programování, který je třikrát větší než model použitý v ostatních algoritmech a navíc obsahuje hodně omezení (vznikaly zaokrouhlovací chyby). Pro MAX $4$-CUT a MAX $5$-CUT vyšel nejlépe algoritmus~\ref{alg:n-max-k-cut-2}, který je jen navrhnut v závěru článku \cite{newman} a není pro něj znám aproximační poměr. U~kapacitní verze byly porovnány dva přístupy: algoritmus založený na lokálním prohledávání a náš algoritmus využívající semidefinitní programování. Oba algoritmy dávaly srovnatelné výsledky, ale rozdíl byl hlavně v době běhu, kde jasně vyhrává algoritmus využívající lokální prohledávání. Algoritmus založený na semidefinitní programování dával lepší výsledky, když součet kapacit jednotlivých množin byl větší než počet vrcholů grafu. 

Dále by mohla být věnována pozornost analýze algoritmu~\ref{alg:n-max-k-cut-2}, o kterém není známé nic.

Celá práce i s implementacemi je dostupná na \url{https://github.com/c0n73x7/D1PL0MK4}.