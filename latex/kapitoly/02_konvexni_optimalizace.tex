\chapter{Konvexní optimalizace}

\section{Obecná podmíněná úloha}

\begin{equation}\label{eq:constrained_problem}
    \begin{split}
        &\min\ f(x) \\
        &g_i(x) \leq 0, i = 1, \dots, m \\
        &h_i(x) = 0, i = 1, \dots, p
    \end{split}
\end{equation}

Hledáme $x \in \mathbb{R}^n$, které minimalizuje $f(x)$, vzhledem k omezením $g_i(x)$ a $h_i(x)$. Proměnné $x$ říkáme \textbf{optimalizační proměnná}, funkci $f(x)$ říkáme \textbf{cenová} nebo \textbf{účelová funkce}. Výrazy $g_i(x) \leq 0$ jsou \textbf{omezení typu nerovnosti} a $h_i(x) = 0$ jsou \textbf{omezení typu rovnosti}. Pokud $m = p = 0$ problém~\ref{eq:constrained_problem} je \textbf{neomezený}, jinak je \textbf{omezený}.

\textbf{Definiční obor} $\mathcal{D}$ úlohy~\ref{eq:constrained_problem} je
$$
    \mathcal{D} = \bigcap_{i=1}^m \textbf{dom}\ g_i \cap \bigcap_{i=1}^p \textbf{dom}\ h_i.
$$
Říkáme, že bod $x \in \mathcal{D}$ je \textbf{přípustný}, jestliže splňuje všechna omezení $g_i(x) \leq 0$ a $h_i(x) = 0$. Úloha~\ref{eq:constrained_problem} je \textbf{přípustná}, jestliže existuje alespoň jeden bod $x \in \mathcal{D}$, který je přípustný. Množina všech přípustných bodů $x \in \mathcal{D}$ se nazývá \textbf{přípustná množina}.

\textbf{Optimální hodnota} $x^*$ úlohy~\ref{eq:constrained_problem} je definována jako
$$
    x^* = \left\{ f(x) \mid g_i(x) \leq 0, i = 1, \dots, m, h_i(x) = 0, i = 1, \dots, p \right\}.
$$

\section{Konvexní podmíněná úloha}

\begin{equation}\label{eq:convex_problem}
    \begin{split}
        &\min\ f(x) \\
        &g_i(x) \leq 0, i = 1, \dots, m \\
        &a_i^Tx = b_i, i = 1, \dots, p
    \end{split}
\end{equation}

Oproti obecné úloze~\ref{eq:constrained_problem} jsou funkce $f(x), g_i(x)$ konvexní a funkce $h_i(x) = a_i^Tx - b_i$ jsou afinní. Přípustná množina takové úlohy je konvexní množinou.

\section{Lagrangeova dualita}

Mějme úlohu~\ref{eq:constrained_problem} s $\mathcal{D} \neq 0$. Zobrazení $L:\ \mathbb{R}^n \times \mathbb{R}^m \times \mathbb{R}^p \rightarrow \mathbb{R}$ takové, že
\begin{equation}
    L(x, \lambda, \mu) = f(x) + \sum_{i=1}^m \lambda_i g_i(x) + \sum_{i=1}^p \mu_i h_i(x)
\end{equation}
se nazývá \textbf{Lagrangeova funkce}. Definiční obor $\textbf{dom}\ L = \mathcal{D} \times \mathbb{R}^m \times \mathbb{R}^p$. Vektory $\lambda$ a $\mu$ nazýváme \textbf{duální proměnné} a prvkům těchto vektorů říkáme \textbf{Lagrangeovy multiplikátory}. Dále definujeme \textbf{duální funkci} $d:\ \mathbb{R}^m \times \mathbb{R}^p \rightarrow \mathbb{R}$ jako infimum Lagrangeovy funkce $L$ přes všechna $x \in \mathcal{D}$. Tedy
\begin{equation}
    d(\lambda, \mu) = \inf_{x \in \mathcal{D}} L(x, \lambda, \mu) = \inf_{x \in \mathcal{D}} \left( f(x) + \sum_{i=1}^m \lambda_i g_i(x) + \sum_{i=1}^p \mu_i h_i(x) \right).
\end{equation}

Poznamenejme, že duální funkce je konkávní bez ohledu na to, zda je úloha konvexní a je-li $L$ zdola neomezená v proměnné $x$, potom duální funkce nabývá hodnoty $-\infty$.

\subsection{Dolní odhad na $x^*$}\label{s:lower_bound}

Nechť $\tilde{x}$ je přípustný bod. Pro $\lambda \geq 0$ je
$$
    \sum_{i=1}^m \lambda_i g_i(\tilde{x}) + \sum_{i=1}^p \mu_i h_i(\tilde{x}) \leq 0.
$$
Potom pro Lagrangeovu funkci můžeme psát
$$
    L(\tilde{x}, \lambda, \mu) = f(\tilde{x}) + \sum_{i=1}^m \lambda_i g_i(\tilde{x}) + \sum_{i=1}^p \mu_i h_i(\tilde{x}) \leq f(\tilde{x}).
$$
A tedy pro duální funkci platí
$$
    d(\lambda, \mu) = \inf_{x \in \mathcal{D}} L(x, \lambda, \mu) \leq L(\tilde{x}, \lambda, \mu) \leq f(\tilde{x}).
$$

\subsection{Duální úloha}

V části~\ref{s:lower_bound} jsme si ukázali, že duální funkce dává dolní odhad na optimální hodnotu $x^*$ úlohy~\ref{eq:constrained_problem}. Stále jsme si ale neřekli, jaký je nejlepší dolní odhad, který pomocí duální funkce jsme schopni dostat. To nás dostává k následující optimalizační úloze.
\begin{equation}\label{eq:dual_problem}
    \begin{split}
        &\max d(\lambda, \mu) \\
        &\lambda \geq 0
    \end{split}
\end{equation}

Úloze~\ref{eq:dual_problem} se říká \textbf{Lagrangeova duální úloha} příslušná k úloze~\ref{eq:constrained_problem}, kterou nazýváme \textbf{primární úlohou}.

\subsection{Slabá dualita}

Optimální řešení Lagrangeovy duální úlohy označíme $d^*$, které je už z definice nejlepší dolní odhad na optimální řešení primární úlohy $p^*$. Tato nerovnost platí i pokud primární úloha není konvexní. Této nerovnosti říkáme \textbf{slabá dualita}. Rozdíl optimálních řešení $p^* - d^*$ označujeme jako \textbf{optimální dualitní rozdíl} primární úlohy. Poznamenejme, že optimální dualitní rozdíl je vždy nezáporný.


\subsection{Silná dualita a Slaterova podmínka}

Pokud je optimální dualitní rozdíl $p^* - d^* = 0$, pak říkáme, že platí silná dualita. Silná dualita obecně neplatí, ale pro primární úlohu, která splňuje nějaké další podmínky to možné je. Těmto podmínkám se říká \textbf{podmínky kvalifikace omezení}. Jednou takovou je \textbf{Slaterova podmínka}:
$$
    \exists x \in \textbf{relint}\ \mathcal{D}:\ f_i(x) < 0, i = 1, \dots, m, Ax = b.
$$

Bodu $x \in \mathcal{D}$, který splňuje Slaterovu podmínku, říkáme, že je \textbf{striktně přípustný}, protože omezení typu nerovnosti jsou ostré. Pokud jsou některé funkce $f_i$ afinní, můžeme Slaterovu podmínku modifikovat. Nechť tedy $f_1, \dots, f_k, k \leq m$, jsou afinní funkce. Potom \textbf{modifikovaná Slaterova podmínka} má tvar:
$$
    \exists x \in \textbf{relint}\ \mathcal{D}:\ f_i(x) \leq 0, i = 1, \dots, k, f_i(x) < 0, i = k+1, \dots, m, Ax = b.
$$

Pro úlohu~\ref{eq:convex_problem} platí následující věta.
\begin{vt}[Slaterova]
    Nechť primární úloha je konvexní a platí (modifikovaná) Slaterova podmínka, potom $p^* = d^*$.
\end{vt}


\section*{TODO}
\textbf{Použití duální úlohy:} obecnou primární úlohu je těžké vyřešit, ale duální úloha je vždy konvexní, tak vyřeším tu a mám alespoň dolní odhad na primární úlohu
