\chapter*{Úvod}
\addcontentsline{toc}{chapter}{Úvod}

Matematické programování (optimalizace) se zabývá hledáním optimálních řešení matematických modelů. Modely, ve kterých se vyskytují pouze lineární funkce se zabývá lineární programování. Začátky lineárního programování jsou spojeny s druhou světovou válkou. George Dantzig, který měl na starosti vývoj logistických plánu na straně amerického letectva, v roce 1947 vymyslel Simplexovou metodu. S podobným konceptem přišel už dříve v roce 1939 Leonid Kantorovič, ale na jeho práci bohužel nikdo nenavázal. Další slavná jména, která stojí za zmínku v souvislosti s lineárním programováním jsou John von Neumann, Albert Tucker, Harold Kuhn a spousta dalších. Relativně nová oblast optimalizace se nazývá semidefinitní programování, kterou dobře charakterizuje název článku z roku 1981 s názvem \textit{Linear Programming with Matrix Variables} od Cravena a Monda. Pokud máme optimalizační problém, ve kterém řešení jsou vyjádřena pomocí diskrétních proměnných, pak hovoříme o problému kombinatorické optimalizace. K řešení těchto problémů se v posledních cca 30 letech rozšířilo semidefinitní programování, které se využívá např. u Shannonovy kapacity grafu, studia řezů, problému obchodního cestujícího a dalších. Pro další historické informace související s optimalizací doporučuji zdroj \cite{history}, ze kterého bylo čerpáno.