\chapter{Lineární programování}

\section{Formulace úlohy}

Úlohou lineárního programování rozumíme minimalizaci nebo maximalizaci lineární \textbf{účelové funkce} vzhledem k lineárním \textbf{omezením}, kde tato omezení jsou dána soustavou lineární rovnic a nerovnic. Úlohu lineárního programování lze formulovat v několika tvarech, které se liší zadáním omezení. Úloha ve \textbf{standardním tvaru} má svá omezení dána soustavou lineárních rovnic $Ax = b$. Tedy
\begin{equation}
    \min \left\{ c^T x \mid Ax = b, x \geq 0 \right\}, \tag{LP-P}
    \label{eq:LP-P}
\end{equation}
kde $A \in \mathbb{R}^{m \times n}$, $b \in \mathbb{R}^n$, $x \in \mathbb{R}^n$ a $c \in \mathbb{R}^n$. \textbf{Přípustná množina řešení} je průnikem affiního prostoru, který je definován soustavou rovnic $Ax = b$ a \textbf{nezáporného ortantu}, tj. množiny $\left\{ x \in \mathbb{R}^n \mid x_i \geq 0, i = 1, \dots, n \right\}$. Obě tyto množiny jsou konvexní a tedy i jejich průnik je rovněž konvexní množina. Dále, protože přípustnou množinu máme popsanou soustavou konečně mnoha lineárních rovnic a nerovnic, geometricky se na úlohu \ref{eq:LP-P} můžeme koukat jako na minimalizaci lineární funkce přes polyedr, který je definován touto soustavou. 

\section{Dualita}

\section{Komplementární skluzovost}