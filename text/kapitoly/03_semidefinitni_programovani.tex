\chapter{Semidefinitní programování}

Na semidefinitní programování se můžeme koukat jako na zobecnění lineárního programování, kde proměnné jsou symetrické matice. Jedná se tedy o optimalizaci lineární funkce vzhledem k tzv. lineárním maticovým nerovnostem.

\section{Vsuvka z lineární algebry}

\subsection*{Pozitivně definitní a semidefinitní matice}

Pracujeme s reálnými symetrickými maticemi $S = S^T$. Ty mají všechna vlastní čísla reálná a některé z nich mají zajímavou vlastnost, že všechna jejich vlastní čísla jsou kladná. Takovým maticím říkáme, že jsou pozitivně definitní. Alternativní definicí je, že matice $S$ je pozitivně definitní, jestliže $x^TSx > 0$ pro všechny nenulové vektory $x$ (tzv. energy test).

\begin{pr}
$$
    x^T S x = 
    \begin{bmatrix}
        x_1 & x_2
    \end{bmatrix}
    \begin{bmatrix}
        2 & 4 \\
        4 & 9
    \end{bmatrix}
    \begin{bmatrix}
        x_1 \\
        x_2
    \end{bmatrix} =
    2 x_1^2 + 8 x_1 x_2 + 9 x_2^2
$$
Je pro všechny $x$ nenulové $x^TSx > 0$? Ano, protože můžeme výraz přepsat na součet čtverců:
$$
    x^TSx = 2 x_1^2 + 8 x_1 x_2 + 9 x_2^2 = 2 (x_1 + 2 x_2)^2 + x_2^2.
$$
TODO: obrázek kyblíčku
\end{pr}


\begin{vt}
    $S = S^T$ je pozitivně definitní, jestliže lze napsat jako $S = A^T A$ pro nějakou matici $A$, která má lineárně nezávislé sloupce.
\end{vt}
\begin{proof}
    \begin{equation}
    \label{eq:tmp}
        x^TSx = x^TA^TAx = (Ax)^T(Ax) = \lVert Ax \rVert^2 \geq 0
    \end{equation}
    $\lVert Ax \rVert^2 > 0$ jestliže sloupce matice $A$ jsou lineárně nezávislé
\end{proof}

\begin{pr}
    $$
        S = 
        \begin{bmatrix}
            2 & 3 & 4 \\
            3 & 5 & 7 \\
            4 & 7 & 10
        \end{bmatrix} =
        \begin{bmatrix}
            1 & 1 \\
            1 & 2 \\
            1 & 3
        \end{bmatrix}
        \begin{bmatrix}
            1 & 1 & 1 \\
            1 & 2 & 3
        \end{bmatrix} =
        AA^T
    $$
    $A$ má lineárně závislé sloupečky, tj. $S$ není pozitivně definitní
\end{pr}

\begin{vt}
    $S = S^T$ je pozitivně definitní, jestliže všechny hlavní minory $S$ jsou kladné.
\end{vt}

\begin{pr}
    TODO
\end{pr}

\begin{vt}
    $S = S^T$ je pozitivně definitní, jestliže jsou všechny pivoty při eliminaci kladné.
\end{vt}

\begin{pr}
    TODO
\end{pr}

\noindent\textbf{??pozitivně semidefinitní matice -- energy test $\geq 0$, $S=A^TA$ povoleny LZ sloupečky, hlavní minory nezáporné, pivoty nezáporné??}

\subsection*{Pozitivně semidefinitní kužel}

Množinu všech symetrických matic značíme $S^n$, množinu všech pozitivně semidefinitních matic značíme $S_+^n$ a množinu všech pozitivně definitních matic značíme $S_{++}^n$. 

\begin{vt}
    Množina $S_+^n$ tvoří konvexní kužel.
\end{vt}

\begin{proof}
    $\Theta_1, \Theta_2 \geq 0$, $A, B \in S_+^n$
    $$
        x^T \left( \Theta_1 A + \Theta_2 B \right) x = x^T \Theta_1 A x + x^T \Theta_2 B x \geq 0.
    $$
\end{proof}

Množině $S_+^n$ se říká pozitivně semidefinitní kužel. \textbf{??chci ukázat i uzavřenost a další vlastnosti, jak se česky řekne proper cone??}

\subsection*{Spektraedry}

Definujeme tzv. L\"{o}wnerovo částečné uspořádání
$$
    A \succeq B \iff A - B \in S_+^n.
$$

\begin{df}
    Lineární maticová nerovnost (LMI) je ve tvaru
    $$
        A_0 + \sum_{i=1}^n A_i x_i \succeq 0,
    $$
    kde $A_i \in S^n$.
\end{df}

\begin{df}
    Říkáme, že $S \subset \mathbb{R}^n$ je spektraedr, jestliže lze napsat ve tvaru
    $$
        S = \left\{ (x_1, \dots, x_m) \in \mathbb{R}^m \mid A_0 + \sum_{i=1}^m A_i x_i \succeq 0 \right\}
    $$
    pro nějaké symetrické matice $A_0, \dots, A_m \in S^n$.
\end{df}

Spektraedr je tedy množina, která je definována konečným počtem LMI. Můžeme si všimnout analogie s definicí polyedru, který je přípustnou množinu pro lineární program. Podobně spektraedr je přípustnou množinou pro semidefinitní program.

Geometricky je spektraedr definován jako průnik pozitivně semidefinitního kuželu $S_+^n$ a afinního podprostoru $\textbf{span}\left\{ A_1, \dots, A_m \right\}$ posunutého do $A_0$.

Spektraedry jsou uzavřené množiny, neboť LMI je ekvivalentní nekonečně mnoha skalárních nerovností ve tvaru $v^T(A_0 + \sum_{i=1}^m A_ix_i)v \geq 0$, jednu pro každou hodnotu $v \in \mathbb{R}^n$.

Vždy můžeme několik LMI \uv{scucnout} do jedné. Stačí zvolit matice $A_i$ blokově diagonální. Odtud vídíme, že polyedr je speciálním případem spektraedru. Polyedr bude mít všechny matice $A_i$ diagonální.

\begin{pr}
    eliptická křivka
\end{pr}


\section*{WIP}

Stopa matice $A \in R^{n \times n}$ je $\textbf{Tr}(A) = \sum_{i=1}^n A_{ii}$.

\begin{vt}
$S, T$ jsou pozitivně definitní $\implies$ $S + T$ je pozitivně definitní
\end{vt}

\begin{proof}
$x^T (S+T) x = x^T S x + x^T T x > 0$
\end{proof}

\subsection*{Formulace primární úlohy}


\subsection*{Dualita}
\noindent Lagrangovo funkce a dualita konvexního programování;

\noindent z toho dualitu semidefinitního programování;

\noindent rozdíl oproti lineárnímu programování;

\noindent příklad

\subsection*{Relaxace}
\noindent vektorové programování

%spektraedr, formulace, dualita, semidefinitní kužel, duální kužel, ...
%vektorové programování a jeho ekvivalence, formulace úloh