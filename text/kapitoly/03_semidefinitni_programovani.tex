\chapter{Semidefinitní programování}

Na semidefinitní programování se můžeme koukat jako na zobecnění lineárního programování, kde proměnné jsou symetrické matice. Jedná se tedy o optimalizaci lineární funkce vzhledem k tzv. lineárním maticovým nerovnostem.

\section{Vsuvka z lineární algebry}

Množinu všech symetrických matic značíme $S^n$. Říkáme, že $A \in S^n$ je
\begin{enumerate}
    \item pozitivně definitní, jestliže $\forall x:\ x^TAx >0$,
    \item pozitivně semidefinitní, jestliže $\forall x \neq 0:\ x^TAx \geq 0$.
\end{enumerate}

Množinu všech pozitivně semidefinitních matic značíme $S_+^n$ a množinu všech pozitivně definitních matic značíme $S_{++}^n$. 

\begin{vt}
    Množina $S_+^n$ tvoří konvexní kužel.
\end{vt}

\begin{proof}
    $\Theta_1, \Theta_2 \geq 0$, $A, B \in S_+^n$
    $$
        x^T \left( \Theta_1 A + \Theta_2 B \right) x = x^T \Theta_1 A x + x^T \Theta_2 B x \geq 0.
    $$
\end{proof}

Množině $S_+^n$ se říká pozitivně semidefinitní kužel. \textbf{??chci ukázat i uzavřenost a další vlastnosti??}

Definujeme tzv. L\"{o}wnerovo částečné uspořádání
$$
    A \succeq B \iff A - B \in S_+^n.
$$

Stopa matice $A \in R^{n \times n}$ je $\textbf{Tr}(A) = \sum_{i=1}^n A_{ii}$.

\begin{df}
    Lineární maticová nerovnost (LMI) je ve tvaru
    $$
        A_0 + \sum_{i=1}^n A_i x_i \succeq 0,
    $$
    kde $A_i \in S^n$.
\end{df}

\begin{df}
    Říkáme, že $S \subset \mathbb{R}^n$ je spektraedr, jestliže lze napsat ve tvaru
    $$
        S = \left\{ (x_1, \dots, x_m) \in \mathbb{R}^m \mid A_0 + \sum_{i=1}^m A_i x_i \succeq 0 \right\}
    $$
    pro nějaké symetrické matice $A_0, \dots, A_m \in S^n$.
\end{df}

Spektraedr je tedy množina, která je definována konečným počtem LMI. Můžeme si všimnout analogie s definicí polyedru, který je přípustnou množinu pro lineární program. Podobně spektraedr je přípustnou množinou pro semidefinitní program.

Geometricky je spektraedr definován jako průnik pozitivně semidefinitního kuželu $S_+^n$ a afinního podprostoru $\textbf{span}\left\{ A_1, \dots, A_m \right\}$ posunutého do $A_0$.

Spektraedry jsou uzavřené množiny, neboť LMI je ekvivalentní nekonečně mnoha skalárních nerovností ve tvaru $v^T(A_0 + \sum_{i=1}^m A_ix_i)v \geq 0$, jednu pro každou hodnotu $v \in \mathbb{R}^n$.

Vždy můžeme několik LMI \uv{scucnout} do jedné. Stačí zvolit matice $A_i$ blokově diagonální. Odtud vídíme, že polyedr je speciálním případem spektraedru. Polyedr bude mít všechny matice $A_i$ diagonální.

\begin{pr}
    eliptická křivka
\end{pr}

\section*{WIP}

kritéria pozitivní definitnosti a semidefinitnosti matic

\begin{enumerate}
    \item $S$ je symetrická pozitivně definitní
    \item $\lambda_i > 0, i=1, \dots, n$
    \item energie $x^TSx > 0$ (příklad ve 2D)
    \item $S = A^TA$ (sloupce $A$ jsou lineárně nezávislé)
    \item všechny hlavní minory jsou $> 0$
    \item všechny pivoty při eliminaci jsou $> 0$
\end{enumerate}


\begin{vt}
$S, T$ jsou pozitivně definitní $\implies$ $S + T$ je pozitivně definitní
\end{vt}

\begin{proof}
$x^T (S+T) x = x^T S x + x^T T x > 0$
\end{proof}

\subsection*{Formulace primární úlohy}


\subsection*{Dualita}
\noindent Lagrangovo funkce a dualita konvexního programování;

\noindent z toho dualitu semidefinitního programování;

\noindent rozdíl oproti lineárnímu programování;

\noindent příklad

\subsection*{Relaxace}
\noindent vektorové programování

%spektraedr, formulace, dualita, semidefinitní kužel, duální kužel, ...
%vektorové programování a jeho ekvivalence, formulace úloh