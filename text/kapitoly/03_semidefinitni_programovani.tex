\chapter{Semidefinitní programování}

zatím je struktura.. 

\section{Formulace úlohy}
\noindent příklad

\section{Vsuvka o semidefinitních maticích}

kritéria pozitivní definitnosti a semidefinitnosti matic

\begin{enumerate}
    \item $S$ je pozitivně definitní
    \item $\lambda_i > 0, i=1, \dots, n$
    \item energie $x^TSx > 0$ (příklad ve 2D)
    \item $S = A^TA$ (sloupce $A$ jsou lineárně nezávislé)
    \item všechny hlavní minory jsou $> 0$
    \item všechny pivoty při eliminaci jsou $> 0$
\end{enumerate}


\begin{vt}
$S, T$ jsou pozitivně definitní $\implies$ $S + T$ je pozitivně definitní
\end{vt}

\begin{proof}
$x^T (S+T) x = x^T S x + x^T T x > 0$
\end{proof}


\section{Semidefinitní kužel}



\section{Dualita}
\noindent Lagrangovo funkce a dualita konvexního programování;

\noindent z toho dualitu semidefinitního programování;

\noindent rozdíl oproti lineárnímu programování;

\noindent příklad

\section{Relaxace}
\noindent vektorové programování

%spektraedr, formulace, dualita, semidefinitní kužel, duální kužel, ...
%vektorové programování a jeho ekvivalence, formulace úloh