\chapter{Konvexní optimalizace}

\section{Podmíněná optimalizace}

\begin{equation}\label{eq:constrained_optimization_problem}
    \begin{split}
        &\min\ f(x) \\
        &g_i(x) \leq 0, i = 1, \dots, m \\
        &h_i(x) = 0, i = 1, \dots, p
    \end{split}
\end{equation}

Hledáme $x \in \mathbb{R}^n$, které minimalizuje $f(x)$, vzhledem k omezením $g_i(x)$ a $h_i(x)$. Proměnné $x$ říkáme \textbf{optimalizační proměnná}, funkci $f(x)$ říkáme \textbf{cenová} nebo \textbf{účelová funkce}. Výrazy $g_i(x) \leq 0$ jsou \textbf{omezení typu nerovnosti} a $h_i(x) = 0$ jsou \textbf{omezení typu rovnosti}. Pokud $m = p = 0$ problém~\ref{eq:constrained_optimization_problem} je \textbf{neomezený}, jinak je \textbf{omezený}.

\textbf{Definiční obor} $\mathcal{D}$ úlohy~\ref{eq:constrained_optimization_problem} je
$$
    \mathcal{D} = \bigcap_{i=1}^m \textbf{dom}\ g_i \cap \bigcap_{i=1}^p \textbf{dom}\ h_i.
$$
Říkáme, že bod $x \in \mathcal{D}$ je \textbf{přípustný}, jestliže splňuje všechna omezení $g_i(x) \leq 0$ a $h_i(x) = 0$. Úloha~\ref{eq_constrained_optimization_problem} je \textbf{přípustná}, jestliže existuje alespoň jeden bod $x \in \mathcal{D}$, který je přípustný. Množina všech přípustných bodů $x \in \mathcal{D}$ se nazývá \textbf{přípustná množina}.

\textbf{Optimální hodnota} $x^*$ úlohy~\ref{eq:constrained_optimization_problem} je definována jako
$$
    x^* = \left\{ f(x) \mid g_i(x) \leq 0, i = 1, \dots, m, h_i(x) = 0, i = 1, \dots, p \right\}.
$$

\section{Konvexní optimalizace}

\begin{equation}\label{eq:convex_optimization_problem}
    \begin{split}
        &\min\ f(x) \\
        &g_i(x) \leq 0, i = 1, \dots, m \\
        &a_i^Tx = b_i, i = 1, \dots, p
    \end{split}
\end{equation}

Oproti obecné úloze~\ref{eq:constrained_optimization_problem} jsou funkce $f(x), g_i(x)$ konvexní a funkce $h_i(x) = a_i^Tx - b_i$ jsou afinní. Přípustná množina takové úlohy je konvexní množinou.

\section{Lagrangeova dualita}

