\part{Teorie}

\chapter{Základní geometrické pojmy}
%úsečka, afinní množina, konvexní množina, ...

Mějme dva body $x_1, x_2 \in \mathbb{R}^n$ takové, že $x_1 \neq x_2$ a parametr $\theta \in \mathbb{R}^n$. Potom výraz
\begin{equation}
    y = \theta x_1 + (1 - \theta) x_2
    \label{line}
\end{equation}
popisuje přímku procházející body $x_1$ a $x_2$. Pro $\theta = 0$ dostáváme bod $x_2$ a pro $\theta = 1$ bod $x_1$. Omezíme-li tedy $\theta$ na interval $\langle 0, 1 \rangle$, dostaneme úsečku s koncovými body $x_1$ a $x_2$. Výraz \ref{line} lze přepsat do tvaru
\begin{equation}
    y = x_2 + \theta (x_1 - x_2),
\end{equation}
který můžeme interpretovat jako součet počátečního bodu $x_2$ a nějakého násobku směrového vektoru $x_1 - x_2$.

Říkáme, že množina $C \subseteq \mathbb{R}^n$ je afinní, jestliže přímka procházející libovolnými dvěma různými body z $C$ leží v $C$.



\chapter{Lineární programování}
%polyedr, formulace, dualita, ...
%příklad s řešením (dieta?)

\chapter{Semidefinitní programování}
%spektraedr, formulace, dualita, semidefinitní kužel, duální kužel, ...
%vektorové programování a jeho ekvivalence, formulace úloh

\chapter{Kuželové programování}
%jen stručně pro úplnost
